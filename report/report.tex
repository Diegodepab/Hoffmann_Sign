\documentclass{bmcart}

%%%%%%%%%%%%%%%%%%%%%%%%%%%%%%%%%%%%%%%%%%%%%%
%%                                          %%
%% CARGA DE PAQUETES DE LATEX               %%
%%                                          %%
%%%%%%%%%%%%%%%%%%%%%%%%%%%%%%%%%%%%%%%%%%%%%%

%para las tablas
\usepackage{float}

%%% Load packages
\usepackage{amsthm,amsmath}
\usepackage{graphicx}
%\RequirePackage[numbers]{natbib}
%\RequirePackage{hyperref}
\usepackage{hyperref}
\usepackage[utf8]{inputenc} %unicode support
%\usepackage[applemac]{inputenc} %applemac support if unicode package fails
%\usepackage[latin1]{inputenc} %UNIX support if unicode package fails
\usepackage{tabularx} 
\usepackage{array} 
\usepackage{caption} 
\usepackage{hyperref}

%%%%%%%%%%%%%%%%%%%%%%%%%%%%%%%%%%%%%%%%%%%%%%
%%                                          %%
%% COMIENZO DEL DOCUMENTO                   %%
%%                                          %%
%%%%%%%%%%%%%%%%%%%%%%%%%%%%%%%%%%%%%%%%%%%%%%

\begin{document}

	\begin{frontmatter}
	
		\begin{fmbox}
			\dochead{Research}
			
			%%%%%%%%%%%%%%%%%%%%%%%%%%%%%%%%%%%%%%%%%%%%%%
			%% INTRODUCIR TITULO PROYECTO               %%
			%%%%%%%%%%%%%%%%%%%%%%%%%%%%%%%%%%%%%%%%%%%%%%
			
			\title{Hoffmann's sign}
			
			%%%%%%%%%%%%%%%%%%%%%%%%%%%%%%%%%%%%%%%%%%%%%%
			%% AUTORES. METER UNA ENTRADA AUTHOR        %%
			%% POR PERSONA                              %%
			%%%%%%%%%%%%%%%%%%%%%%%%%%%%%%%%%%%%%%%%%%%%%%
			
			\author[
			  addressref={aff1},                   % ESTA LINEA SE COPIA IGUAL PARA CADA AUTOR
			  corref={aff1},                       % ESTA LINEA SOLO DEBE TENERLA EL COORDINADOR DEL GRUPO
			  email={depablodiego@uma.es}   % VUESTRO CORREO ACTIVO
			]{\inits{J.E.}\fnm{Diego} \snm{De Pablo}} % inits: INICIALES DE AUTOR, fnm: NOMBRE DE AUTOR, snm: APELLIDOS DE AUTOR
			\author[
			  addressref={aff1},
			  email={alexsilva@uma.es}
			]{\inits{A.S.R.}\fnm{Alejandro} \snm{Silva Rodríguez}}\author[
			addressref={aff1},
			email={0610948742@uma.es} 
			]{\inits{J.I.S.M.}\fnm{Juan Ignacio} \snm{Soriano Muñoz}}\author[
			addressref={aff1},
			email={martacuevas@uma.es} 
			]{\inits{J.I.S.M.}\fnm{Marta} \snm{Cuevas Rodríguez}}
			
			%%%%%%%%%%%%%%%%%%%%%%%%%%%%%%%%%%%%%%%%%%%%%%
			%% AFILIACION. NO TOCAR                     %%
			%%%%%%%%%%%%%%%%%%%%%%%%%%%%%%%%%%%%%%%%%%%%%%
			
			\address[id=aff1]{%                           % unique id
			  \orgdiv{ETSI Informática},             % department, if any
			  \orgname{Universidad de Málaga},          % university, etc
			  \city{Málaga},                              % city
			  \cny{España}                                    % country
			}
		
		\end{fmbox}% comment this for two column layout
		
		\begin{abstractbox}
		
			\begin{abstract} % abstract
			
			El resumen debe ser de un solo párrafo, entre 100 y 130 palabras. No incluir URL ni referencias a figuras o
			esquemas. Las referencias no incluirse en el resumen. Idealmente debe incluir el contexto, el objeFvo el
			trabajo y la principal aportación o novedad deducida del trabajo. Tiempos verbales: resultados y lo que se hace
			en pasado. futuro en cuanto a implicaciones a futura
			
			\end{abstract}
			
			%%%%%%%%%%%%%%%%%%%%%%%%%%%%%%%%%%%%%%%%%%%%%%
			%% PALABRAS CLAVE DEL PROYECTO              %%
			%%%%%%%%%%%%%%%%%%%%%%%%%%%%%%%%%%%%%%%%%%%%%%
			
			\begin{keyword}
			\kwd{Signo de Hoffmann}
			\kwd{Fenotipo}
			\kwd{Cluster}
			\kwd{Análisis de red}
			\end{keyword}
		
		
		\end{abstractbox}
	
	\end{frontmatter}
	%%%%%%%%%%%%%%%%%%%%%%%%%%%%%%%%%
	%% COMIENZO DEL DOCUMENTO REAL %%
	%%%%%%%%%%%%%%%%%%%%%%%%%%%%%%%%%
	
	\section{Introducción}
El signo de Hoffmann es un reflejo muscular que se produce al percutir suavemente el lecho ungueal del dedo medio o índice, como se muestra en la figura \ref{fig:Hoffmann_sign}, produciéndose un movimiento de flexión involuntario del pulgar cuando el examinador hace girar la uña del dedo medio hacia abajo. Fue propuesto por primera vez por Johann Hoffmann, un neurólogo alemán, a finales del siglo XIX, y fue descrito por primera vez por Hans Curschmann, uno de sus asistentes, en 1911 \cite{BENDHEIM}. El signo de Hoffmann también ha sido denominado de diferentes formas, como 'reflejo digital', 'reflejo de chasquido', 'signo de Tromner' y 'signo de Jakobson' \cite{glaser2001cervical}.

\begin{figure}[h!]
	\includegraphics[width=0.35\textwidth]{figures/Kabir_Hoffmann__Sign.jpg}
	\caption{Signo de Hoffmann. Este diagrama muestra un signo de Hoffmann positivo, una parte estándar del examen neurológico común. Contribución de R Kabir, MD}
	\label{fig:Hoffmann_sign}
\end{figure}

Se ha utilizado en la práctica clínica durante aproximadamente cien años como una herramienta para detectar alteraciones en las vías corticoespinales, las cuales conectan la corteza cerebral con la médula espinal. Estudios realizados en la década de 1930 evaluaron la incidencia del signo en estudiantes universitarios sanos, encontrando una incidencia del 2\% y 1.63\% \cite{echols1936hoffmann} \cite{fay1933clinical}, respectivamente, aunque solo se incluyeron sujetos masculinos \cite{glaser2001cervical}. Este hallazgo clínico ha sido útil en la detección de mielopatía cervical espondilótica temprana \cite{denno1991early}, como lo propusieron Denno y Meadows al describir el signo de Hoffmann 'dinámico', una variante de la prueba con flexiones activas del cuello \cite{glaser2001cervical}.

Es importante destacar que el signo de Hoffmann es un fenotipo y no una enfermedad en sí, ya que se ha descubierto que hasta el 3\% de la población presenta un signo de Hoffmann positivo sin que haya compresión de la médula. Este reflejo está asociado a 12 enfermedades diferentes \cite{whitney}.

El signo de Hoffmann ha sido identificado en una serie de enfermedades neurodegenerativas y trastornos del tracto corticoespinal, muchas de ellas caracterizadas por alteraciones motoras progresivas. Entre estas patologías se encuentran diversas formas de paraplejía espástica hereditaria, que son un grupo clínicamente y genéticamente heterogéneo de trastornos neurológicos, caracterizados principalmente por espasticidad progresiva y, a menudo, pérdida del sentido de la vibración en los miembros inferiores \cite{Esteves2014}, tanto autosómica dominante como recesiva. Por ejemplo, la paraplejía espástica 9A, de herencia autosómica dominante (Online Mendelian Inheritance in Man, OMIM:601162) \cite{10.1093/brain/awv143}, y las formas recesivas como la paraplejía espástica 72 (OMIM:615625), asociadas con disfunción motora grave.

Enfermedades neurodegenerativas más conocidas, como la esclerosis lateral amiotrófica (ORPHA:803), también muestran una asociación con el signo de Hoffmann, debido a la degeneración de las motoneuronas superiores \cite{RIANCHO201927}. Diversas formas de ataxias espásticas, relacionadas con la falta de coordinación motora \cite{Pedroso2022}, como la ataxia espástica 9 (OMIM:618438) y 10 (OMIM:620666), completan el espectro de condiciones en las que este reflejo patológico se manifiesta.

A nivel molecular, diversos genes han sido asociados con condiciones que incluyen este signo, reflejo que indica alteraciones en los tractos corticoespinales. Entre estos genes destacan SOD1, TARDBP, UBQLN2 y NEK1, todos vinculados a la esclerosis lateral amiotrófica (ELA). Las mutaciones en SOD1 \cite{zhao2022g41d}, TARDBP \cite{sanchez2022atypical} y UBQLN2 \cite{teyssou:hal-03001781} afectan las motoneuronas superiores, contribuyendo a la aparición de reflejos patológicos como el signo de Hoffmann. Además, NEK1 ha sido recientemente asociado con formas hereditarias de ELA \cite{mann2023NEK1}, lo que refuerza su implicación en el deterioro de las vías motoras. Las alteraciones en estos genes provocan una degeneración progresiva de las neuronas motoras, subrayando la relevancia del signo de Hoffmann como un marcador clínico clave en enfermedades neurodegenerativas.


	\section{Objetivos}
\subsection{Objetivo General}

Explorar las interacciones entre genes y proteínas asociadas al Signo de Hoffmann, utilizando bases de datos bioinformáticas y herramientas de análisis de redes para identificar posibles grupos funcionales y patrones de interacción relevantes.

\subsection{Objetivos Específicos}
\begin{enumerate}
	\item Identificar genes asociados al signo de Hoffmann mediante la utilización de la Human Phenotype Ontology (HPO) y otras bases de datos relevantes.
	\item Construir una red de interacciones proteína-proteína (PPI) basada en los genes obtenidos, utilizando StringDB para analizar las interacciones de las proteínas codificadas por estos genes.
	\item Aplicar algoritmos de análisis de redes para calcular métricas topológicas y determinar características clave de la red.
	\item Aplicar clustering en la red de interacción para identificar grupos de genes o proteínas que presenten una alta conectividad.
	\item Determinar las principales funciones biológicas y vías metabólicas en las que están involucrados los genes identificados mediante enriquecimiento funcional.
\end{enumerate}

\section{Materiales y Herramientas}

\subsection{Bases de datos}

\subsubsection{Human Phenotype Ontology (HPO)}
  Es una base de datos que estandariza los fenotipos clínicos humanos y vincula genes y enfermedades a cada fenotipo obteniendo las interacciones genéticas.\cite{gargano2024}. 
\subsubsection{StringDB}
La base de datos STRING (\textit{Search Tool for the Retrieval of Interacting Genes/Proteins}) es una base de datos que reúne datos experimentales, predicciones y literatura para informar sobre interacciones proteicas. \cite{szklarczyk2019}.

\subsection{Lenguajes de Programación}

\subsubsection{R}
El lenguaje de programación \textit{R} específicamente la versión 4.3.3. Es un lenguaje para exploración estadística y creación de gráficos. Es flexible, ampliable con paquetes y de código abierto bajo el proyecto GNU. \cite{chan2018}.


\paragraph{Manipulación y Visualización de Datos:}
\begin{itemize}
	\item \textbf{tidyverse}: Conjunto de paquetes (incluyendo \textit{dplyr} y \textit{ggplot2}) que permiten manipular y graficar datos. \textit{dplyr} facilita el manejo de grandes datasets, mientras que \textit{ggplot2} permite generar gráficos de alta calidad \cite{Wickham2019}.
\end{itemize}

\paragraph{Análisis Bioinformático:}
\begin{itemize}
	\item \textbf{Bioconductor}: Conjunto de paquetes especializados en el análisis de datos genómicos. \cite{Huber2015}.
	\item \textbf{iGraph para R}: Paquete para R que contiene herramientas para la manipulación y análisis de redes. Además, incluye funciones para medir propiedades globales como modularidad y densidad \cite{Csardi2006}.
\end{itemize}


\subsubsection{Python}
El lenguaje de programación \textit{Python} es versátil y de alto nivel, usado en diversas aplicaciones.  Es interpretado, por lo que no requiere compilación previa, que permite el uso de librerías y APIs para diversas funciones. A continuación, se describen las librerías específicas empleadas en este estudio:

\paragraph{Análisis de Redes y Grafos:}
\begin{itemize}
	\item \textbf{NetworkX}: Complemento de iGraph para la visualización y análisis interactivo de grafos, permitiendo inspeccionar la estructura y propiedades de redes de interacción proteína-proteína (PPI) \cite{hagberg2008}.
\end{itemize}

\paragraph{Extracción y Manipulación de Datos:}
\begin{itemize}
	\item \textbf{Requests}: Necesario para realizar consultas a APIs como HPO y StringDB, a fin de recuperar datos de interacciones. \cite{Requests2020}.
	\item \textbf{Pandas}: Herramienta de manipulación de datos, utilizada para estructurar y limpiar datos previos a su análisis en redes.\cite{McKinney2010}.
\end{itemize}





\section{Métodos}

\subsection{Flujo de trabajo}

\begin{figure}[h!]
	\includegraphics[width=.95\textwidth]{figures/workflow.png}
	\caption{Flujo de trabajo para obtener analisis de enriquecimiento mejorado del fenotipo signo de Hoffmann}
	\label{fig:workflow}
\end{figure}

\subsection{Obtención de genes relacionados con signo de Hoffmann}

Para obtener los genes relacionados con el signo de Hoffmann (HP:0031993), se utilizó la API de la Ontología de Fenotipos Humanos (HPO). El recurso consultado contiene nuestro termino HPO y el endpoint de genes (https://ontology.jax.org/api/
network/annotation/HP:0031993/download/gene). Mediante una solicitud HTTP, se extrajo y almacenó esta información en archivos con formato .tsv, facilitando así su posterior análisis. 


\subsection{Obtención de red de interacciones de genes}

Para el análisis de interacciones génicas, se utilizó una red de interacciones obtenida de la base de datos STRING. Las interacciones de proteínas humanas (\textit{Homo sapiens}, ID taxonómico: 9606) fueron descargadas de el endpoint de descargas de STRING (https://stringdb-downloads.org/download/prot ein.links.v12.0/9606.protein.links.v12.0.txt.gz). Solo se consideraron interacciones con una puntuación combinada (\textit{combined score}) mayor o igual a 400.

\newpage

\subsection{Propagación de red}


Para identificar genes adicionales asociados al conjunto de genes iniciales relacionados con el Signo de Hoffmann, se aplicó el algoritmo de propagación de red DIAMOnD \cite{Ghiassian2015}. Se utilizó la red de interacciones génicas como base, y el algoritmo añadió nodos que optimizaban la cercanía topológica al clúster de genes, priorizando aquellos con menor p-valor. En caso de empate, se empleó la suma ponderada de los pesos de las conexiones con los genes del grafo para desempatar, dando preferencia a los nodos con conexiones más relevantes y fuertemente conectados.


\subsection{Análisis de red}

Para el análisis de redes, se utilizó el algoritmo de Louvain \cite{Blondel2008} para detectar módulos funcionales en la red optimizando la partición de la red y maximizando su modularidad. Además, se implementaron métricas topológicas como la centralidad de grado, que mide el número de conexiones directas de cada nodo; la centralidad de intermediación, que identifica nodos clave en el flujo de información; y la centralidad de cercanía, que evalúa la proximidad de un nodo respecto al resto. Estas métricas, calculadas mediante la librería iGraph descrita en los materiales.

\subsection{Análisis de enriquecimiento funcional}

Una vez identificados los clusters, se realizó un análisis de enriquecimiento funcional con la herramienta Enrichr \cite{10.1093/nar/gkad393/1} que compara diferentes fuentes de datos para determinar qué funciones biológicas, rutas metabólicas o procesos celulares están sobrerrepresentados en los grupos de genes detectados.

	\section{Resultados}

Este estudio explora las interacciones génicas asociadas al Signo de Hoffman con el fin de identificar patrones relevantes que contribuyan a la comprensión de su rol en enfermedades neurodegenerativas. Se aplicaron técnicas de propagación de red para añadir genes adicionales y se realizaron análisis topológicos y de enriquecimiento funcional para determinar las funciones biológicas y vías metabólicas implicadas.

\subsection{Red de interacción entre genes}


Se obtienen 46 genes relacionados con el termino del signo de Hoffmann de HPO, a los cuales se le aplica propagación de red mediante DIAMOnD hasta establecer 66 genes (20 genes añadidos).
\\

\textbf{INSERTAR FIGURA DE LA RED DSP DE DIAMOND PERO SIN NADA}\\

\subsection{Propiedades de la red y detección de comunidades}


\textbf{AQUI ES MEJOR QUE SEA EN TABLA}
\\

En el análisis de la red generada a partir de los genes asociados al signo de Hoffman, se identificaron un total de 78 nodos y 799 aristas, lo que evidencia un nivel significativo de interacciones entre los genes estudiados. El grado promedio de la red fue de 20.49, indicando que, en promedio, cada gen está conectado a más de 20 otros genes en la red.

Entre las métricas topológicas evaluadas, la centralidad de cercanía y la centralidad de intermediación destacan como indicadores clave del rol estructural de ciertos nodos en la red. En cuanto a la centralidad de cercanía, los genes \textbf{COQ4} y \textbf{COQ7} presentaron los valores más altos, indicando que estos nodos están óptimamente posicionados para acceder rápidamente a otros genes en la red. Otros genes destacados en esta métrica incluyen \textbf{CTNNB1}, \textbf{SMAD3} y \textbf{SMAD4}, que también mostraron valores elevados de cercanía. Por otro lado, la centralidad de intermediación identificó a \textbf{CTNNB1} como el nodo más relevante en términos de flujo de información dentro de la red, seguido de \textbf{FUS}, \textbf{VCP}, \textbf{SQSTM1} y \textbf{TGFB1}, lo que resalta su importancia en la conectividad global y en el enlace entre distintas comunidades.

La modularidad de la red fue de 0.42, lo que sugiere una estructura modular moderada con comunidades bien definidas. Se identificaron seis comunidades principales, siendo la más grande la comunidad 1, con 37 nodos, seguida por la comunidad 3, con 32 nodos. Las otras comunidades presentaron tamaños significativamente menores, con 3, 2, 2 y 2 nodos respectivamente. Esto indica que la red está compuesta por un núcleo principal de genes altamente interconectados, junto con comunidades más pequeñas y especializadas.

Finalmente, la densidad de la red fue de 0.27, reflejando una conectividad moderada dentro de la red, compatible con redes biológicas donde no todos los genes están directamente conectados pero presentan interacciones relevantes en función de sus roles funcionales. Estos resultados proporcionan una base sólida para realizar análisis funcionales y explorar el papel de las comunidades detectadas en el contexto del signo de Hoffman.


\subsection{Análisis de enriquecimiento funcional}

Se obtenieron resultados significativos para los clusters uno y tres, que son los que más número de genes albergan. En las figuras \ref{tb:c1_t1}, \ref{tb:c1_t2}, \ref{tb:c1_t3}, \ref{tb:c1_t4}, \ref{tb:c2_t1}, \ref{tb:c2_t2}, \ref{tb:c2_t3}, \ref{tb:c2_t4} se muestran los resultados más significativos por cada clúster y categoría.
\begin{table}[H]
	\centering
	\caption{Análisis de Enriquecimiento - GO Biological Process - Cluster 1}
	\label{tb:c1_t1}
	\begin{tabular}{|p{4cm}|p{4cm}|p{3cm}|}
		\hline
		\textbf{Término} & \textbf{Genes} & \textbf{p-value} \\ \hline
		Golgi to plasma membrane transport (GO:0006893) & RAB10, EXOC8, EXOC6B, EXOC4, EXOC6, EXOC5, EXOC2, EXOC1 & 7.33e-17 \\ \hline
		cilium organization (GO:0044782) & ARF4, EXOC8, EXOC7, RAB3IP, ASAP1, RAB11A, EXOC4, EXOC3, EXOC6, EXOC5, RAB11FIP3, RAB8A, EXOC2, EXOC1 & 1.53e-23 \\ \hline
		plasma membrane bounded cell projection assembly (GO:0120031) & ARF4, EXOC8, EXOC7, RAB3IP, ASAP1, RAB11A, EXOC4, EXOC3, EXOC6, EXOC5, RAB11FIP3, RAB8A, EXOC2, EXOC1 & 2.60e-22 \\ \hline
	\end{tabular}
\end{table}

\begin{table}[H]
	\centering
	\caption{Análisis de Enriquecimiento - GO Cellular Component - Cluster 1}
	\label{tb:c1_t2}
	\begin{tabular}{|p{4cm}|p{4cm}|p{3cm}|}
		\hline
		\textbf{Término} & \textbf{Genes} & \textbf{p-value} \\ \hline
		insulin-responsive compartment (GO:0032593) & RAB10, MYO5A & 3.41e-05 \\ \hline
		recycling endosome (GO:0055037) & RAB10, MYO5A, RAB11FIP3, RAB11A, RAB8A & 2.65e-07 \\ \hline
		recycling endosome membrane (GO:0055038) & RAB11FIP3, RAB11A, RAB8A & 2.55e-05 \\ \hline
	\end{tabular}
\end{table}

\begin{table}[H]
	\centering
	\caption{Análisis de Enriquecimiento - GO Molecular Function - Cluster 1}
	\label{tb:c1_t3}
	\begin{tabular}{|p{4cm}|p{4cm}|p{3cm}|}
		\hline
		\textbf{Término} & \textbf{Genes} & \textbf{p-value} \\ \hline
		myosin V binding (GO:0031489) & RAB10, RAB11A, RAB8A & 3.86e-07 \\ \hline
		myosin binding (GO:0017022) & RAB10, RALA, RAB11A, RAB8A & 2.58e-07 \\ \hline
		small GTPase binding (GO:0031267) & EXOC8, EXOC4, MYO5A, EXOC5, RAB11FIP3, RAB8A, EXOC2 & 2.45e-10 \\ \hline
	\end{tabular}
\end{table}

\begin{table}[H]
	\centering
	\caption{Análisis de Enriquecimiento - KEGG Pathways - Cluster 1}
	\label{tb:c1_t4}
	\begin{tabular}{|p{4cm}|p{4cm}|p{3cm}|}
		\hline
		\textbf{Término} & \textbf{Genes} & \textbf{p-value} \\ \hline
		Endocytosis & RAB10, ASAP1, RAB11FIP3, ARF5, RAB11A, RAB8A & 1.04e-07 \\ \hline
		Pancreatic cancer & RALA, IKBKG & 0.00252 \\ \hline
		Pancreatic secretion & RAB11A, RAB8A & 0.00426 \\ \hline
	\end{tabular}
\end{table}


\begin{table}[H]
	\centering
	\caption{Análisis de Enriquecimiento - Procesos Biológicos (GO:BP) - Cluster 3}
	\label{tb:c2_t1}
	\begin{tabular}{|p{4cm}|p{4cm}|p{3cm}|}
		\hline
		\textbf{Término} & \textbf{Genes} & \textbf{p-value} \\ \hline
		Positive regulation of ATP biosynthetic process (GO:2001171) & VCP, TREM2, PPARGC1A & 3.10e-07 \\ \hline
		Positive regulation of purine nucleotide biosynthetic process (GO:1900373) & VCP, TREM2, PPARGC1A & 1.05e-06 \\ \hline
		Regulation of ATP biosynthetic process (GO:2001169) & VCP, TREM2, PPARGC1A & 2.05e-06 \\ \hline
	\end{tabular}
\end{table}

\begin{table}[H]
	\centering
	\caption{Análisis de Enriquecimiento - Componentes Celulares (GO:CC) - Cluster 3}
	\label{tb:c2_t2}
	\begin{tabular}{|p{4cm}|p{4cm}|p{3cm}|}
		\hline
		\textbf{Término} & \textbf{Genes} & \textbf{p-value} \\ \hline
		Intracellular non-membrane-bounded organelle (GO:0043232) & FIG4, GLE1, VCP, TAF15, DCTN1, ANXA11, ANG, NEFH & 3.72e-04 \\ \hline
		Mitochondrial intermembrane space (GO:0005758) & CHCHD10, SOD1 & 3.88e-03 \\ \hline
		Organelle envelope lumen (GO:0031970) & CHCHD10, SOD1 & 4.70e-03 \\ \hline
	\end{tabular}
\end{table}


\begin{table}[H]
	\centering
	\caption{Análisis de Enriquecimiento - Funciones Moleculares (GO:MF) - Cluster 3}
	\label{tb:c2_t3}
	\begin{tabular}{|p{4cm}|p{4cm}|p{3cm}|}
		\hline
		\textbf{Término} & \textbf{Genes} & \textbf{p-value} \\ \hline
		Polyubiquitin modification-dependent protein binding (GO:0031593) & VCP, SQSTM1, OPTN, UBQLN2 & 1.28e-06 \\ \hline
		miRNA binding (GO:0035198) & MATR3, HNRNPA1 & 1.05e-03 \\ \hline
		Regulatory RNA binding (GO:0061980) & MATR3, HNRNPA1 & 1.86e-03 \\ \hline
	\end{tabular}
\end{table}

\begin{table}[H]
	\centering
	\caption{Análisis de Enriquecimiento - Rutas KEGG - Cluster 3}
	\label{tb:c2_t4}
	\begin{tabular}{|p{4cm}|p{4cm}|p{3cm}|}
		\hline
		\textbf{Término} & \textbf{Genes} & \textbf{p-value} \\ \hline
		Amyotrophic lateral sclerosis (ALS) & NEFH, SOD1, PRPH & 7.35e-05 \\ \hline
		Mitophagy & TBK1, SQSTM1, OPTN & 1.52e-04 \\ \hline
		Huntington disease & DCTN1, PPARGC1A, SOD1 & 3.57e-03 \\ \hline
	\end{tabular}
\end{table}







	\section{Discusión}
\subsection{Cluster 1}

\subsubsection{GO Biological Process}

Procesos como el transporte del Golgi a la membrana plasmática (\textit{GO:0006893}) y la organización del cilio (\textit{GO:0044782}) están relacionados con el tráfico vesicular y la señalización celular, esenciales para la dinámica axonal. Estas funciones son cruciales para mantener la integridad de los circuitos neuronales, lo cual podría ser relevante en los reflejos anormales observados en el signo de Hoffman \cite{nachury_ciliary_2010}. Asimismo, el ensamblaje de proyecciones celulares (\textit{GO:0120031}) apunta a un papel en la plasticidad estructural de las neuronas.

\subsubsection{GO Cellular Component}

El endosoma de reciclaje (\textit{GO:0055037}) es clave para la regulación de receptores de membrana, afectando directamente la plasticidad sináptica y el tráfico intracelular neuronal \cite{goldenring_endosome_2019}. Estos procesos son esenciales para la modulación de la actividad sináptica y podrían estar alterados en condiciones de hiperreflexia. Además, el compartimento sensible a insulina (\textit{GO:0032593}) podría influir indirectamente en la regulación del tráfico intracelular en neuronas.

\subsubsection{GO Molecular Function}

La unión a pequeñas GTPasas (\textit{GO:0031267}) y a miosina (\textit{GO:0017022}) subraya la importancia del citoesqueleto en la transmisión neuronal. Alteraciones en estas interacciones podrían comprometer la estructura axonal, lo que es coherente con neuropatías motoras y la disfunción de los reflejos profundos \cite{feiguin_axonal_transport_2001}.

\subsubsection{KEGG Pathways}

La vía de la endocitosis resalta como fundamental en el transporte vesicular y la comunicación neuronal \cite{conner_endocytosis_2003}. Disfunciones en esta ruta pueden alterar la excitabilidad neuronal y contribuir a fenómenos como el signo de Hoffman.




\subsection{Cluster 3}

En la discusión, consideramos apropiado respaldar nuestros resultados citando artículos que vinculan los términos con la neurodegeneración, la cual, como hemos mencionado anteriormente, guarda una estrecha relación con el signo de Hoffman.

\subsubsection{Positive regulation of ATP biosynthetic process - GO Biological Process}
El artículo \cite{Bonvento2017} explica que la regulación positiva del proceso de biosíntesis de ATP está directamente relacionada con la neurodegeneración, ya que las neuronas dependen de una producción eficiente de ATP para mantener los gradientes iónicos esenciales para la transmisión sináptica. La disfunción mitocondrial reduce esta capacidad, lo que provoca la acumulación de daño celular característico de las enfermedades neurodegenerativas. Además, las alteraciones en este proceso intensifican el estrés oxidativo, acelerando el deterioro neuronal y contribuyendo al avance de la enfermedad. 

\subsubsection{Endosome to lysosome transport via multivesicular body sorting pathway - GO Biological Process}
En el artículo \cite{Mulligan2023}, se describe cómo las disfunciones en esta vía de transporte llevan a la acumulación de proteínas mal plegadas y otros desechos celulares en las neuronas. Esta acumulación es una característica común en muchas enfermedades neurodegenerativas, como el Alzheimer y el Parkinson. El estudio utiliza técnicas avanzadas de microscopía para investigar estas interacciones y proporciona un conjunto de enfoques combinatorios para la imagen fija y la imagen en vivo de células, lo que permite una comprensión más profunda de los procesos intracelulares dinámicos.


\subsubsection{Maintenance of synapse structure - GO Biological Process}
En el artículo \cite{Batool2019}, se describe cómo el mantenimiento de la estructura sináptica (GO:0099558) es crucial para la función adecuada del sistema nervioso, y cómo las disfunciones en este proceso pueden llevar a trastornos neurodegenerativos. Se destaca que la formación y el mantenimiento de sinapsis no son estáticos, sino que cambian constantemente para satisfacer las necesidades conductuales del organismo. El estudio utiliza técnicas avanzadas para investigar estos procesos y proporciona una comprensión más profunda de los mecanismos celulares y moleculares involucrados en la formación y el mantenimiento de sinapsis, así como su relación con trastornos neurodegenerativos.

\subsubsection{Mitochondrial intermembrane space - GO Cellular Component}

En el artículo \cite{Kathiresan2024}, se describe cómo las disfunciones en el espacio intermembrana mitocondrial pueden contribuir a la patogénesis de trastornos neurodegenerativos como el Alzheimer, el Parkinson y la enfermedad de Huntington. Se destaca que las mutaciones en el ADN mitocondrial y las alteraciones en la dinámica mitocondrial pueden llevar a una producción de energía comprometida y un aumento del estrés oxidativo, lo que resulta en daño neuronal y muerte celular. El estudio también explora estrategias terapéuticas potenciales dirigidas a la disfunción mitocondrial, incluyendo terapias específicas para mitocondrias y antioxidantes.

\subsubsection{Cytoplasmic stress granule - GO Cellular Component}

En el artículo \cite{10.1093/nar/gkae655}, se describe cómo la formación de gránulos de estrés citoplasmáticos puede mitigar la neurodegeneración. Los gránulos de estrés son complejos de ARN y proteínas que se forman en respuesta a condiciones de estrés celular y juegan un papel crucial en la regulación de la traducción y la supervivencia celular. El estudio utilizó el modelo de la proteína nsP3 del alfavirus para reducir la formación de gránulos de estrés y observó que, en modelos de ataxia, esclerosis lateral amiotrófica y demencia frontotemporal, la reducción de estos gránulos exacerbó los fenotipos de la enfermedad. Esto sugiere que los gránulos de estrés pueden tener un papel protector en las enfermedades neurodegenerativas

En el artículo \cite{PMID:34248597}, se describe cómo los gránulos de estrés citoplasmáticos juegan un papel crucial en la respuesta celular al estrés y su relación con enfermedades neurodegenerativas. Los gránulos de estrés son estructuras sin membrana que se forman en respuesta a condiciones de estrés y ayudan a regular la traducción de ARN y la supervivencia celular. El estudio destaca que la formación y dinámica de estos gránulos están implicadas en la patogénesis de enfermedades como la esclerosis lateral amiotrófica y la demencia frontotemporal, sugiriendo que los gránulos de estrés pueden actuar como precursores de agregados patológicos en estas enfermedades.

\subsubsection{Polyubiquitin modification-dependent protein binding - GO Molecular Function}

En el artículo \cite{Schmidt2021}, se describe cómo la unión de proteínas dependiente de la modificación por poliubiquitina (GO:0031593) juega un papel crucial en la señalización celular y la degradación de proteínas en enfermedades neurodegenerativas. Se destaca que las vías de degradación, como el sistema ubiquitina-proteasoma y la vía autofagia-lisosoma, dependen de la modificación de proteínas con ubiquitina para eliminar proteínas mal plegadas y mantener la salud celular. El estudio también explora cómo la disfunción en estas vías puede llevar a la acumulación de agregados proteicos neurotóxicos, contribuyendo a la patogénesis de enfermedades como el Alzheimer, el Parkinson y la esclerosis lateral amiotrófica.

\subsubsection{K63-linked polyubiquitin modification-dependent protein binding - GO Molecular Function}
En el artículo \cite{10.1093/hmg/ddm320}, se describe cómo la ubiquitinación dependiente de la modificación por poliubiquitina enlazada en K63 (GO:0070530) promueve la formación y la eliminación autofágica de inclusiones proteicas asociadas con enfermedades neurodegenerativas. Se destaca que la ubiquitinación en K63 facilita la acumulación de proteínas y la formación de inclusiones intracelulares, incluso en ausencia de deterioro del proteasoma. Además, esta modificación específica de ubiquitina ayuda a definir el destino de las proteínas para su eliminación a través de la autofagia, proporcionando una ruta mecanística novedosa para el ciclo de vida de los cuerpos de inclusión y ofreciendo potenciales enfoques terapéuticos para tratar trastornos neurodegenerativos.

\subsubsection{Amyotrophic lateral sclerosis (ALS) - KEGG Pathways}

\subsubsection{Mitophagy - KEGG Pathways}

En el artículo \cite{Zhang2022}, se describe cómo la mitofagia, el proceso selectivo de degradación de mitocondrias dañadas o disfuncionales, es crucial para mantener la salud mitocondrial y prevenir la acumulación de mitocondrias defectuosas, lo que es fundamental en la patogénesis de enfermedades neurodegenerativas como el Alzheimer, el Parkinson y la esclerosis lateral amiotrófica. El estudio destaca los mecanismos moleculares que regulan la mitofagia, incluyendo las vías canónicas y no canónicas, y cómo la disfunción en estos procesos puede llevar a la acumulación de mitocondrias dañadas, contribuyendo a la neurodegeneración.
	\section{Conclusiones}

\subsection{Conclusión General}

El análisis identificó interacciones clave entre genes y proteínas relacionadas con el Signo de Hoffmann, detectando grupos funcionales y patrones de interacción potencialmente implicados en procesos neuronales y enfermedades neurodegenerativas, lo que contribuye al entendimiento de sus bases moleculares.

\subsection{Conclusiones Específicas}

\begin{enumerate}
	\item Se identificaron genes relacionados con el Signo de Hoffmann utilizando la Human Phenotype Ontology (HPO) y otras bases de datos relevantes, además de genes que podrían estar indirectamente relacionadas, facilitando la observación de patrones funcionales que no eran evidentes.
	
	\item Se construyó con éxito una red de interacciones proteína-proteína (PPI) destacando nodos relevantes con alta conectividad.
	
	\item Mediante el cálculo de métricas topológicas, se identificaron proteínas clave dentro de la red con potencial importancia en la regulación de procesos neurobiológicos relacionados con la hiperreflexia y la neurodegeneración.
	
	\item La aplicación de algoritmos de clustering permitió identificar grupos funcionales altamente conectados dentro de la red de interacciones. Estos clusters mostraron correlaciones con procesos biológicos relevantes, como la endocitosis y la regulación del citoesqueleto.
	
	\item El análisis de enriquecimiento funcional confirmó que los genes identificados están involucrados en funciones biológicas críticas, como la regulación de la biosíntesis de ATP y el transporte intracelular, y en vías relacionadas con la neurodegeneración.
\end{enumerate}

\section{Líneas futuras de investigación}

Los resultados de este estudio abren varias líneas de investigación futura sobre los mecanismos moleculares del Signo de Hoffmann y su relación con enfermedades neurodegenerativas. Aunque el análisis funcional y de redes proporciona una base sólida, se recomienda realizar estudios experimentales para validar las interacciones y funciones propuestas, así como integrar nuevos datos para un análisis más completo.

\begin{itemize}
	\item Validación experimental de las interacciones proteína-proteína identificadas en modelos celulares o animales.
	\item Exploración de las modificaciones postraduccionales de proteínas clave involucradas en los procesos identificados.
	\item Ampliación del análisis a otras patologías relacionadas con hiperreflexia para identificar posibles conexiones comunes.
	\item Análisis longitudinal de la expresión de los genes identificados en diferentes etapas de la neurodegeneración.
	\item Integración de datos ómicos adicionales (transcriptómica, epigenómica) para un análisis más exhaustivo de las redes moleculares implicadas.
\end{itemize}


	
	
	%%%%%%%%%%%%%%%%%%%%%%%%%%%%%%%%%%%%%%%%%%%%%%
	%% OTRA INFORMACIÓN                         %%
	%%%%%%%%%%%%%%%%%%%%%%%%%%%%%%%%%%%%%%%%%%%%%%
	
	\begin{backmatter}
	
		\section*{Abreviaciones}%% if any
		\begin{itemize}
			\item \textbf{HPO}: Human Phenotype Ontology, En español: La ontología del fenotipo humano.
			\item \textbf{STRING}: Search Tool for the Retrieval of Interacting Genes/Proteins En español: Herramienta de búsqueda para la recuperación de genes/proteínas que interactúan
			\item \textbf{OMIM}: Online Mendelian Inheritance in Man, En español: Herencia mendeliana en línea en el hombre.
			\item \textbf{ORPHA}:Online Database of Rare Diseases and Orphan Drugs.
			\item \textbf{ELA}: esclerosis lateral amiotrófica.
			\item \textbf{KEGG}: (Kyoto Encyclopedia of Genes and Genomes) 
		\end{itemize}
		
		\section*{Disponibilidad de datos y materiales}%% if any
			Puedes encontrar más información en el \href{https://github.com/Diegodepab/project_template}{repositorio de github}
			 
			
		\section*{Contribución de los autores}
			En este trabajo hubo implicación en todas las áreas por parte de los autores, incluyendo la redacción y corrección del trabajo pero se destaca el desempeño individual en lo que código respecta. D. D: en la propagación de redes, A. S: En el enriquecimiento funcional además de automatización del código, M. C: en el análisis de red y J. S: en la búsqueda de módulos funcionales
			
		
		
		%%%%%%%%%%%%%%%%%%%%%%%%%%%%%%%%%%%%%%%%%%%%%%%%%%%%%%%%%%%%%%%%%%%%%%%%%%%%%%%%%%%%%%%%
		%% BIBLIOGRAFIA: no teneis que tocar nada, solo sustituir el archivo bibliography.bib %%
		%% por el que hayais generado vosotros                                                %%
		%%%%%%%%%%%%%%%%%%%%%%%%%%%%%%%%%%%%%%%%%%%%%%%%%%%%%%%%%%%%%%%%%%%%%%%%%%%%%%%%%%%%%%%%
		
		\bibliographystyle{bmc-mathphys} % Style BST file (bmc-mathphys, vancouver, spbasic).
		\bibliography{bibliography}      % Bibliography file (usually '*.bib' )
		
	\end{backmatter}
\end{document}
