\section{Resultados}

Este estudio explora las interacciones génicas asociadas al Signo de Hoffman con el fin de identificar patrones relevantes que contribuyan a la comprensión de su rol en enfermedades neurodegenerativas. Se aplicaron técnicas de propagación de red para añadir genes adicionales y se realizaron análisis topológicos y de enriquecimiento funcional para determinar las funciones biológicas y vías metabólicas implicadas.

\subsection{Red de interacción entre genes}


Se obtienen 46 genes relacionados con el termino del signo de Hoffmann de HPO, a los cuales se le aplica propagación de red mediante DIAMOnD hasta establecer 66 genes (20 genes añadidos).
\\

\textbf{INSERTAR FIGURA DE LA RED DSP DE DIAMOND PERO SIN NADA}\\

\subsection{Propiedades de la red y detección de comunidades}


\textbf{AQUI ES MEJOR QUE SEA EN TABLA}
\\

En el análisis de la red generada a partir de los genes asociados al signo de Hoffman, se identificaron un total de 78 nodos y 799 aristas, lo que evidencia un nivel significativo de interacciones entre los genes estudiados. El grado promedio de la red fue de 20.49, indicando que, en promedio, cada gen está conectado a más de 20 otros genes en la red.

Entre las métricas topológicas evaluadas, la centralidad de cercanía y la centralidad de intermediación destacan como indicadores clave del rol estructural de ciertos nodos en la red. En cuanto a la centralidad de cercanía, los genes \textbf{COQ4} y \textbf{COQ7} presentaron los valores más altos, indicando que estos nodos están óptimamente posicionados para acceder rápidamente a otros genes en la red. Otros genes destacados en esta métrica incluyen \textbf{CTNNB1}, \textbf{SMAD3} y \textbf{SMAD4}, que también mostraron valores elevados de cercanía. Por otro lado, la centralidad de intermediación identificó a \textbf{CTNNB1} como el nodo más relevante en términos de flujo de información dentro de la red, seguido de \textbf{FUS}, \textbf{VCP}, \textbf{SQSTM1} y \textbf{TGFB1}, lo que resalta su importancia en la conectividad global y en el enlace entre distintas comunidades.

La modularidad de la red fue de 0.42, lo que sugiere una estructura modular moderada con comunidades bien definidas. Se identificaron seis comunidades principales, siendo la más grande la comunidad 1, con 37 nodos, seguida por la comunidad 3, con 32 nodos. Las otras comunidades presentaron tamaños significativamente menores, con 3, 2, 2 y 2 nodos respectivamente. Esto indica que la red está compuesta por un núcleo principal de genes altamente interconectados, junto con comunidades más pequeñas y especializadas.

Finalmente, la densidad de la red fue de 0.27, reflejando una conectividad moderada dentro de la red, compatible con redes biológicas donde no todos los genes están directamente conectados pero presentan interacciones relevantes en función de sus roles funcionales. Estos resultados proporcionan una base sólida para realizar análisis funcionales y explorar el papel de las comunidades detectadas en el contexto del signo de Hoffman.


\subsection{Análisis de enriquecimiento funcional}

Se obtenieron resultados significativos para los clusters uno y tres, que son los que más número de genes albergan. En las figuras \ref{tb:c1_t1}, \ref{tb:c1_t2}, \ref{tb:c1_t3}, \ref{tb:c1_t4}, \ref{tb:c2_t1}, \ref{tb:c2_t2}, \ref{tb:c2_t3}, \ref{tb:c2_t4} se muestran los resultados más significativos por cada clúster y categoría.
\begin{table}[H]
	\centering
	\caption{Análisis de Enriquecimiento - GO Biological Process - Cluster 1}
	\label{tb:c1_t1}
	\begin{tabular}{|p{4cm}|p{4cm}|p{3cm}|}
		\hline
		\textbf{Término} & \textbf{Genes} & \textbf{p-value} \\ \hline
		Golgi to plasma membrane transport (GO:0006893) & RAB10, EXOC8, EXOC6B, EXOC4, EXOC6, EXOC5, EXOC2, EXOC1 & 7.33e-17 \\ \hline
		cilium organization (GO:0044782) & ARF4, EXOC8, EXOC7, RAB3IP, ASAP1, RAB11A, EXOC4, EXOC3, EXOC6, EXOC5, RAB11FIP3, RAB8A, EXOC2, EXOC1 & 1.53e-23 \\ \hline
		plasma membrane bounded cell projection assembly (GO:0120031) & ARF4, EXOC8, EXOC7, RAB3IP, ASAP1, RAB11A, EXOC4, EXOC3, EXOC6, EXOC5, RAB11FIP3, RAB8A, EXOC2, EXOC1 & 2.60e-22 \\ \hline
	\end{tabular}
\end{table}

\begin{table}[H]
	\centering
	\caption{Análisis de Enriquecimiento - GO Cellular Component - Cluster 1}
	\label{tb:c1_t2}
	\begin{tabular}{|p{4cm}|p{4cm}|p{3cm}|}
		\hline
		\textbf{Término} & \textbf{Genes} & \textbf{p-value} \\ \hline
		insulin-responsive compartment (GO:0032593) & RAB10, MYO5A & 3.41e-05 \\ \hline
		recycling endosome (GO:0055037) & RAB10, MYO5A, RAB11FIP3, RAB11A, RAB8A & 2.65e-07 \\ \hline
		recycling endosome membrane (GO:0055038) & RAB11FIP3, RAB11A, RAB8A & 2.55e-05 \\ \hline
	\end{tabular}
\end{table}

\begin{table}[H]
	\centering
	\caption{Análisis de Enriquecimiento - GO Molecular Function - Cluster 1}
	\label{tb:c1_t3}
	\begin{tabular}{|p{4cm}|p{4cm}|p{3cm}|}
		\hline
		\textbf{Término} & \textbf{Genes} & \textbf{p-value} \\ \hline
		myosin V binding (GO:0031489) & RAB10, RAB11A, RAB8A & 3.86e-07 \\ \hline
		myosin binding (GO:0017022) & RAB10, RALA, RAB11A, RAB8A & 2.58e-07 \\ \hline
		small GTPase binding (GO:0031267) & EXOC8, EXOC4, MYO5A, EXOC5, RAB11FIP3, RAB8A, EXOC2 & 2.45e-10 \\ \hline
	\end{tabular}
\end{table}

\begin{table}[H]
	\centering
	\caption{Análisis de Enriquecimiento - KEGG Pathways - Cluster 1}
	\label{tb:c1_t4}
	\begin{tabular}{|p{4cm}|p{4cm}|p{3cm}|}
		\hline
		\textbf{Término} & \textbf{Genes} & \textbf{p-value} \\ \hline
		Endocytosis & RAB10, ASAP1, RAB11FIP3, ARF5, RAB11A, RAB8A & 1.04e-07 \\ \hline
		Pancreatic cancer & RALA, IKBKG & 0.00252 \\ \hline
		Pancreatic secretion & RAB11A, RAB8A & 0.00426 \\ \hline
	\end{tabular}
\end{table}


\begin{table}[H]
	\centering
	\caption{Análisis de Enriquecimiento - Procesos Biológicos (GO:BP) - Cluster 3}
	\label{tb:c2_t1}
	\begin{tabular}{|p{4cm}|p{4cm}|p{3cm}|}
		\hline
		\textbf{Término} & \textbf{Genes} & \textbf{p-value} \\ \hline
		Positive regulation of ATP biosynthetic process (GO:2001171) & VCP, TREM2, PPARGC1A & 3.10e-07 \\ \hline
		Positive regulation of purine nucleotide biosynthetic process (GO:1900373) & VCP, TREM2, PPARGC1A & 1.05e-06 \\ \hline
		Regulation of ATP biosynthetic process (GO:2001169) & VCP, TREM2, PPARGC1A & 2.05e-06 \\ \hline
	\end{tabular}
\end{table}

\begin{table}[H]
	\centering
	\caption{Análisis de Enriquecimiento - Componentes Celulares (GO:CC) - Cluster 3}
	\label{tb:c2_t2}
	\begin{tabular}{|p{4cm}|p{4cm}|p{3cm}|}
		\hline
		\textbf{Término} & \textbf{Genes} & \textbf{p-value} \\ \hline
		Intracellular non-membrane-bounded organelle (GO:0043232) & FIG4, GLE1, VCP, TAF15, DCTN1, ANXA11, ANG, NEFH & 3.72e-04 \\ \hline
		Mitochondrial intermembrane space (GO:0005758) & CHCHD10, SOD1 & 3.88e-03 \\ \hline
		Organelle envelope lumen (GO:0031970) & CHCHD10, SOD1 & 4.70e-03 \\ \hline
	\end{tabular}
\end{table}


\begin{table}[H]
	\centering
	\caption{Análisis de Enriquecimiento - Funciones Moleculares (GO:MF) - Cluster 3}
	\label{tb:c2_t3}
	\begin{tabular}{|p{4cm}|p{4cm}|p{3cm}|}
		\hline
		\textbf{Término} & \textbf{Genes} & \textbf{p-value} \\ \hline
		Polyubiquitin modification-dependent protein binding (GO:0031593) & VCP, SQSTM1, OPTN, UBQLN2 & 1.28e-06 \\ \hline
		miRNA binding (GO:0035198) & MATR3, HNRNPA1 & 1.05e-03 \\ \hline
		Regulatory RNA binding (GO:0061980) & MATR3, HNRNPA1 & 1.86e-03 \\ \hline
	\end{tabular}
\end{table}

\begin{table}[H]
	\centering
	\caption{Análisis de Enriquecimiento - Rutas KEGG - Cluster 3}
	\label{tb:c2_t4}
	\begin{tabular}{|p{4cm}|p{4cm}|p{3cm}|}
		\hline
		\textbf{Término} & \textbf{Genes} & \textbf{p-value} \\ \hline
		Amyotrophic lateral sclerosis (ALS) & NEFH, SOD1, PRPH & 7.35e-05 \\ \hline
		Mitophagy & TBK1, SQSTM1, OPTN & 1.52e-04 \\ \hline
		Huntington disease & DCTN1, PPARGC1A, SOD1 & 3.57e-03 \\ \hline
	\end{tabular}
\end{table}






