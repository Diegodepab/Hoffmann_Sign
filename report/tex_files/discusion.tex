\section{Discusión}
\subsection{Cluster 1}

\subsubsection{GO Biological Process}

Procesos como el transporte del Golgi a la membrana plasmática (\textit{GO:0006893}) y la organización del cilio (\textit{GO:0044782}) están relacionados con el tráfico vesicular y la señalización celular, esenciales para la dinámica axonal. Estas funciones son cruciales para mantener la integridad de los circuitos neuronales, lo cual podría ser relevante en los reflejos anormales observados en el signo de Hoffman \cite{cells11182773}. Asimismo, el ensamblaje de proyecciones celulares (\textit{GO:0120031}) apunta a un papel en la plasticidad estructural de las neuronas.

\subsubsection{GO Cellular Component}

El endosoma de reciclaje (\textit{GO:0055037}) es clave para la regulación de receptores de membrana, afectando directamente la plasticidad sináptica y el tráfico intracelular neuronal \cite{RozesSalvador2020}. Estos procesos son esenciales para la modulación de la actividad sináptica y podrían estar alterados en condiciones de hiperreflexia. Además, el compartimento sensible a insulina (\textit{GO:0032593}) podría influir indirectamente en la regulación del tráfico intracelular en neuronas.

\subsubsection{GO Molecular Function}

La unión a pequeñas GTPasas (\textit{GO:0031267}) y a miosina (\textit{GO:0017022}) subraya la importancia del citoesqueleto en la transmisión neuronal. Alteraciones en estas interacciones podrían comprometer la estructura axonal, lo que es coherente con neuropatías motoras y la disfunción de los reflejos profundos \cite{GUO2020133}.

\subsubsection{KEGG Pathways}

La vía de la endocitosis resalta como fundamental en el transporte vesicular y la comunicación neuronal \cite{Chanaday8209}. Disfunciones en esta ruta pueden alterar la excitabilidad neuronal y contribuir a fenómenos como el signo de Hoffmann.




\subsection{Cluster 3}

En la discusión, consideramos apropiado respaldar nuestros resultados citando artículos que vinculan los términos con la neurodegeneración, la cual, como hemos mencionado anteriormente, guarda una estrecha relación con el signo de Hoffmann.

\subsubsection{Positive regulation of ATP biosynthetic process - GO Biological Process}
El artículo \cite{Bonvento2017} explica que la regulación positiva del proceso de biosíntesis de ATP está directamente relacionada con la neurodegeneración, ya que las neuronas dependen de una producción eficiente de ATP para mantener los gradientes iónicos esenciales para la transmisión sináptica. La disfunción mitocondrial reduce esta capacidad, lo que provoca la acumulación de daño celular característico de las enfermedades neurodegenerativas. Además, las alteraciones en este proceso intensifican el estrés oxidativo, acelerando el deterioro neuronal y contribuyendo al avance de la enfermedad. 

\subsubsection{Endosome to lysosome transport via multivesicular body sorting pathway - GO Biological Process}
En el artículo \cite{Mulligan2023}, se describe cómo las disfunciones en esta vía de transporte llevan a la acumulación de proteínas mal plegadas y otros desechos celulares en las neuronas. Esta acumulación es una característica común en muchas enfermedades neurodegenerativas, como el Alzheimer y el Parkinson. El estudio utiliza técnicas avanzadas de microscopía para investigar estas interacciones y proporciona un conjunto de enfoques combinatorios para la imagen fija y la imagen en vivo de células, lo que permite una comprensión más profunda de los procesos intracelulares dinámicos.


\subsubsection{Maintenance of synapse structure - GO Biological Process}
En el artículo \cite{Batool2019}, se describe cómo el mantenimiento de la estructura sináptica (GO:0099558) es crucial para la función adecuada del sistema nervioso, y cómo las disfunciones en este proceso pueden llevar a trastornos neurodegenerativos. Se destaca que la formación y el mantenimiento de sinapsis no son estáticos, sino que cambian constantemente para satisfacer las necesidades conductuales del organismo. El estudio utiliza técnicas avanzadas para investigar estos procesos y proporciona una comprensión más profunda de los mecanismos celulares y moleculares involucrados en la formación y el mantenimiento de sinapsis, así como su relación con trastornos neurodegenerativos.

\subsubsection{Mitochondrial intermembrane space - GO Cellular Component}

En el artículo \cite{Kathiresan2024}, se describe cómo las disfunciones en el espacio intermembrana mitocondrial pueden contribuir a la patogénesis de trastornos neurodegenerativos como el Alzheimer, el Parkinson y la enfermedad de Huntington. Se destaca que las mutaciones en el ADN mitocondrial y las alteraciones en la dinámica mitocondrial pueden llevar a una producción de energía comprometida y un aumento del estrés oxidativo, lo que resulta en daño neuronal y muerte celular. El estudio también explora estrategias terapéuticas potenciales dirigidas a la disfunción mitocondrial, incluyendo terapias específicas para mitocondrias y antioxidantes.

\subsubsection{Cytoplasmic stress granule - GO Cellular Component}

En el artículo \cite{10.1093/nar/gkae655}, se describe cómo la formación de gránulos de estrés citoplasmáticos puede mitigar la neurodegeneración. Los gránulos de estrés son complejos de ARN y proteínas que se forman en respuesta a condiciones de estrés celular y juegan un papel crucial en la regulación de la traducción y la supervivencia celular. El estudio utilizó el modelo de la proteína nsP3 del alfavirus para reducir la formación de gránulos de estrés y observó que, en modelos de ataxia, esclerosis lateral amiotrófica y demencia frontotemporal, la reducción de estos gránulos exacerbó los fenotipos de la enfermedad. Esto sugiere que los gránulos de estrés pueden tener un papel protector en las enfermedades neurodegenerativas

En el artículo \cite{PMID:34248597}, se describe cómo los gránulos de estrés citoplasmáticos juegan un papel crucial en la respuesta celular al estrés y su relación con enfermedades neurodegenerativas. Los gránulos de estrés son estructuras sin membrana que se forman en respuesta a condiciones de estrés y ayudan a regular la traducción de ARN y la supervivencia celular. El estudio destaca que la formación y dinámica de estos gránulos están implicadas en la patogénesis de enfermedades como la esclerosis lateral amiotrófica y la demencia frontotemporal, sugiriendo que los gránulos de estrés pueden actuar como precursores de agregados patológicos en estas enfermedades.

\subsubsection{Polyubiquitin modification-dependent protein binding - GO Molecular Function}

En el artículo \cite{Schmidt2021}, se describe cómo la unión de proteínas dependiente de la modificación por poliubiquitina (GO:0031593) juega un papel crucial en la señalización celular y la degradación de proteínas en enfermedades neurodegenerativas. Se destaca que las vías de degradación, como el sistema ubiquitina-proteasoma y la vía autofagia-lisosoma, dependen de la modificación de proteínas con ubiquitina para eliminar proteínas mal plegadas y mantener la salud celular. El estudio también explora cómo la disfunción en estas vías puede llevar a la acumulación de agregados proteicos neurotóxicos, contribuyendo a la patogénesis de enfermedades como el Alzheimer, el Parkinson y la esclerosis lateral amiotrófica.

\subsubsection{K63-linked polyubiquitin modification-dependent protein binding - GO Molecular Function}
En el artículo \cite{10.1093/hmg/ddm320}, se describe cómo la ubiquitinación dependiente de la modificación por poliubiquitina enlazada en K63 (GO:0070530) promueve la formación y la eliminación autofágica de inclusiones proteicas asociadas con enfermedades neurodegenerativas. Se destaca que la ubiquitinación en K63 facilita la acumulación de proteínas y la formación de inclusiones intracelulares, incluso en ausencia de deterioro del proteasoma. Además, esta modificación específica de ubiquitina ayuda a definir el destino de las proteínas para su eliminación a través de la autofagia, proporcionando una ruta mecanística novedosa para el ciclo de vida de los cuerpos de inclusión y ofreciendo potenciales enfoques terapéuticos para tratar trastornos neurodegenerativos.

\subsubsection{Amyotrophic lateral sclerosis (ALS) - KEGG Pathways}

El resultado de la búsqueda respalda la relación mencionada en la introducción entre el signo de Hoffmann y las enfermedades neurodegenerativas, como la esclerosis lateral amiotrófica (ELA) \cite{RIANCHO201927}.

\subsubsection{Mitophagy - KEGG Pathways}

En el artículo \cite{Zhang2022}, se describe cómo la mitofagia, el proceso selectivo de degradación de mitocondrias dañadas o disfuncionales, es crucial para mantener la salud mitocondrial y prevenir la acumulación de mitocondrias defectuosas, lo que es fundamental en la patogénesis de enfermedades neurodegenerativas como el Alzheimer, el Parkinson y la esclerosis lateral amiotrófica. El estudio destaca los mecanismos moleculares que regulan la mitofagia, incluyendo las vías canónicas y no canónicas, y cómo la disfunción en estos procesos puede llevar a la acumulación de mitocondrias dañadas, contribuyendo a la neurodegeneración.