\section{Discusión}

Los resultados coinciden con estudios previos que vinculan el signo de Hoffmann con la neurodegeneración. Sin embargo, se amplia esta relación al identificar interacciones clave entre los genes implicados, lo que sugiere una conexión adicional con procesos neuronales asociados a enfermedades neurodegenerativas.


\subsection{Cluster 1}

\subsubsection{GO Biological Process}

Procesos como el transporte del Golgi a la membrana plasmática (\textit{GO:0006893}) y la organización del cilio (\textit{GO:0044782}) están relacionados con el tráfico vesicular y la señalización celular, esenciales para la dinámica axonal. Estas funciones son cruciales para mantener la integridad de los circuitos neuronales, lo cual podría ser relevante en los reflejos anormales observados en el signo de Hoffman \cite{cells11182773}. Asimismo, el ensamblaje de proyecciones celulares (\textit{GO:0120031}) apunta a un papel en la plasticidad estructural de las neuronas.

\subsubsection{GO Cellular Component}

El endosoma de reciclaje (\textit{GO:0055037}) es clave para la regulación de receptores de membrana, afectando directamente la plasticidad sináptica y el tráfico intracelular neuronal \cite{RozesSalvador2020}. Estos procesos son esenciales para la modulación de la actividad sináptica y podrían estar alterados en condiciones de hiperreflexia. Además, el compartimento sensible a insulina (\textit{GO:0032593}) podría influir indirectamente en la regulación del tráfico intracelular en neuronas.

\subsubsection{GO Molecular Function}

La unión a pequeñas GTPasas (\textit{GO:0031267}) y a miosina (\textit{GO:0017022}) subraya la importancia del citoesqueleto en la transmisión neuronal. Alteraciones en estas interacciones podrían comprometer la estructura axonal, lo que es coherente con neuropatías motoras y la disfunción de los reflejos profundos \cite{GUO2020133}.

\subsubsection{KEGG Pathways}

La vía de la endocitosis resalta como fundamental en el transporte vesicular y la comunicación neuronal \cite{Chanaday8209}. Disfunciones en esta ruta pueden alterar la excitabilidad neuronal y contribuir a fenómenos como el signo de Hoffmann.


\subsection{Cluster 3}

\subsubsection{Positive regulation of ATP biosynthetic process - GO Biological Process}
La regulación positiva de la biosíntesis de ATP está estrechamente vinculada a la neurodegeneración, ya que las neuronas requieren una producción eficiente de ATP para mantener los gradientes iónicos necesarios en la transmisión sináptica. La disfunción mitocondrial disminuye esta capacidad, lo que conduce a la acumulación de daño celular típico de las enfermedades neurodegenerativas. Además, las alteraciones en este proceso aumentan el estrés oxidativo, acelerando el deterioro neuronal y el avance de la enfermedad \cite{Bonvento2017}.


\subsubsection{Endosome to lysosome transport via multivesicular body sorting pathway - GO Biological Process}
Las disfunciones en la vía de transporte desde el endosoma al lisosoma resultan en la acumulación de proteínas mal plegadas y otros desechos celulares en las neuronas, un fenómeno común en enfermedades neurodegenerativas como el Alzheimer y el Parkinson. Técnicas avanzadas de microscopía han permitido investigar estas interacciones a nivel celular, proporcionando una comprensión más detallada de los procesos intracelulares dinámicos \cite{Mulligan2023}.

\subsubsection{Maintenance of synapse structure - GO Biological Process}
El mantenimiento de la estructura sináptica (GO:0099558) es fundamental para el correcto funcionamiento del sistema nervioso, y las disfunciones en este proceso pueden conducir a trastornos neurodegenerativos. La formación y el mantenimiento de sinapsis son dinámicos, adaptándose a las necesidades conductuales del organismo. El estudio utiliza técnicas avanzadas para investigar estos procesos y proporciona una comprensión más profunda de los mecanismos celulares y moleculares involucrados en la formación y el mantenimiento de sinapsis, así como su relación con trastornos neurodegenerativos. \cite{Batool2019}.


\subsubsection{Mitochondrial intermembrane space - GO Cellular Component}
Las disfunciones en el espacio intermembrana mitocondrial pueden contribuir a la patogénesis de enfermedades neurodegenerativas como el Alzheimer, Parkinson y Huntington. Las mutaciones en el ADN mitocondrial y las alteraciones en su dinámica afectan la producción de energía y aumentan el estrés oxidativo, lo que conduce a daño neuronal y muerte celular. Además, se están explorando estrategias terapéuticas dirigidas a esta disfunción, como terapias mitocondriales y antioxidantes \cite{Kathiresan2024}.


\subsubsection{Cytoplasmic stress granule - GO Cellular Component}
Los gránulos de estrés citoplasmáticos, complejos de ARN y proteínas que se forman en respuesta a estrés celular, juegan un papel clave en la regulación de la traducción y la supervivencia celular. La reducción de estos gránulos ha mostrado agravar fenotipos en modelos de ataxia, esclerosis lateral amiotrófica (ELA) y demencia frontotemporal, lo que sugiere un papel protector en enfermedades neurodegenerativas \cite{10.1093/nar/gkae655}. Además, estos gránulos están implicados en la patogénesis de enfermedades como la ELA y la demencia frontotemporal, actuando posiblemente como precursores de agregados patológicos \cite{PMID:34248597}.

\subsubsection{Polyubiquitin modification-dependent protein binding - GO Molecular Function}
La unión de proteínas dependiente de la poliubiquitina (GO:0031593) es esencial en la señalización celular y en la degradación de proteínas, particularmente en enfermedades neurodegenerativas. Las vías de degradación, como el sistema ubiquitina-proteasoma y la autofagia-lisosoma, utilizan la ubiquitinación para eliminar proteínas mal plegadas y mantener la salud celular. Disfunciones en estas vías pueden provocar la acumulación de agregados proteicos neurotóxicos, lo que contribuye a la patogénesis de enfermedades como el Alzheimer, el Parkinson y la esclerosis lateral amiotrófica \cite{Schmidt2021}.


\subsubsection{K63-linked polyubiquitin modification-dependent protein binding - GO Molecular Function}
La ubiquitinación dependiente de la modificación por poliubiquitina enlazada en K63 (GO:0070530) facilita la acumulación de proteínas y la formación de inclusiones intracelulares, incluso sin deterioro del proteasoma. Este proceso promueve la eliminación autofágica de inclusiones proteicas, asociadas con enfermedades neurodegenerativas. Además, la modificación en K63 define el destino de las proteínas para su eliminación mediante autofagia, proporcionando una ruta mecanística novedosa para el ciclo de vida de los cuerpos de inclusión y ofreciendo potenciales enfoques terapéuticos para tratar trastornos neurodegenerativos. \cite{10.1093/hmg/ddm320}.


\subsubsection{Amyotrophic lateral sclerosis (ALS) - KEGG Pathways}

El resultado de la búsqueda respalda la relación mencionada en la introducción entre el signo de Hoffmann y las enfermedades neurodegenerativas, como la esclerosis lateral amiotrófica (ELA) \cite{RIANCHO201927}.

\subsubsection{Mitophagy - KEGG Pathways}

La mitofagia, el proceso selectivo de degradación de mitocondrias dañadas o disfuncionales, es crucial para mantener la salud mitocondrial y prevenir la acumulación de mitocondrias defectuosas, lo que es fundamental en la patogénesis de enfermedades neurodegenerativas como el Alzheimer, el Parkinson y la esclerosis lateral amiotrófica\cite{Zhang2022}. El estudio destaca los mecanismos moleculares que regulan la mitofagia, incluyendo las vías canónicas y no canónicas, y cómo la disfunción en estos procesos puede llevar a la acumulación de mitocondrias dañadas, contribuyendo a la neurodegeneración.