\section{Discusión}

\subsection{Cluster 3}

Para la discusión vimos correcto respaldar nuestros resultados basándonos en artículos que relacionan 

\subsubsection{Positive regulation of ATP biosynthetic process - GO Biological Process}
El artículo explica que la regulación positiva del proceso de biosíntesis de ATP está directamente relacionada con la neurodegeneración, ya que las neuronas dependen de una producción eficiente de ATP para mantener los gradientes iónicos esenciales para la transmisión sináptica. La disfunción mitocondrial reduce esta capacidad, lo que provoca la acumulación de daño celular característico de las enfermedades neurodegenerativas. Además, las alteraciones en este proceso intensifican el estrés oxidativo, acelerando el deterioro neuronal y contribuyendo al avance de la enfermedad.

\subsubsection{Endosome to lysosome transport via multivesicular body sorting pathway - GO Biological Process}
En el artículo, se describe cómo el mantenimiento de la estructura sináptica (GO:0099558) es crucial para la función adecuada del sistema nervioso, y cómo las disfunciones en este proceso pueden llevar a trastornos neurodegenerativos. Se destaca que la formación y el mantenimiento de sinapsis no son estáticos, sino que cambian constantemente para satisfacer las necesidades conductuales del organismo. El estudio utiliza técnicas avanzadas para investigar estos procesos y proporciona una comprensión más profunda de los mecanismos celulares y moleculares involucrados en la formación y el mantenimiento de sinapsis, así como su relación con trastornos neurodegenerativos.


\subsubsection{Maintenance of synapse structure - GO Biological Process}