\section{Conclusiones}

\subsection{Conclusión General}

El análisis realizado permitió identificar interacciones clave entre genes y proteínas relacionadas con el Signo de Hoffmann, logrando así cumplir con el objetivo general planteado. A través del uso de bases de datos bioinformáticas y herramientas de análisis de redes, se logró detectar grupos funcionales y patrones de interacción que podrían estar implicados en procesos neuronales y enfermedades neurodegenerativas, contribuyendo al entendimiento de las bases moleculares subyacentes a este signo clínico.

\subsection{Conclusiones Específicas}

\begin{enumerate}
	\item Se identificaron genes relacionados con el Signo de Hoffmann utilizando la Human Phenotype Ontology (HPO) y otras bases de datos relevantes, logrando así cumplir el primer objetivo específico. Los genes seleccionados se vinculan con procesos neuronales clave, como la plasticidad sináptica y la transmisión neuronal.
	
	\item Se construyó con éxito una red de interacciones proteína-proteína (PPI) utilizando StringDB. Esta red permitió visualizar las interacciones entre las proteínas codificadas por los genes identificados, destacando nodos relevantes con alta conectividad, alineándose con el segundo objetivo específico.
	
	\item Mediante el cálculo de métricas topológicas, se identificaron proteínas clave dentro de la red con potencial importancia en la regulación de procesos neurobiológicos relacionados con la hiperreflexia y la neurodegeneración, logrando cumplir con el tercer objetivo específico.
	
	\item La aplicación de algoritmos de clustering permitió identificar grupos funcionales altamente conectados dentro de la red de interacciones. Estos clusters mostraron correlaciones con procesos biológicos relevantes, como la endocitosis y la regulación del citoesqueleto, alcanzando el cuarto objetivo específico.
	
	\item El análisis de enriquecimiento funcional confirmó que los genes identificados están involucrados en funciones biológicas críticas, como la regulación de la biosíntesis de ATP y el transporte intracelular, y en vías relacionadas con la neurodegeneración. Este resultado es coherente con el quinto objetivo específico, ya que proporciona un marco para comprender los mecanismos moleculares subyacentes al Signo de Hoffmann.
\end{enumerate}

\section{Líneas futuras de investigación}

A partir de los resultados obtenidos en este estudio, se abren varias líneas futuras de investigación que permitirán profundizar en los mecanismos moleculares relacionados con el Signo de Hoffmann y su implicación en enfermedades neurodegenerativas. El análisis funcional y de redes realizado proporciona una base sólida, pero existen áreas clave que requieren una exploración más detallada. En particular, se sugiere la realización de estudios experimentales para validar las interacciones y funciones propuestas, así como la integración de nuevos datos que permitan un análisis más robusto.

\begin{itemize}
	\item Validación experimental de las interacciones proteína-proteína identificadas en modelos celulares o animales.
	\item Exploración de las modificaciones postraduccionales de proteínas clave involucradas en los procesos identificados.
	\item Ampliación del análisis a otras patologías relacionadas con hiperreflexia para identificar posibles conexiones comunes.
	\item Análisis longitudinal de la expresión de los genes identificados en diferentes etapas de la neurodegeneración.
	\item Integración de datos ómicos adicionales (transcriptómica, epigenómica) para un análisis más exhaustivo de las redes moleculares implicadas.
\end{itemize}

