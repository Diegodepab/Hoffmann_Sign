\section{Conclusiones}

\subsection{Conclusión General}

El análisis identificó interacciones clave entre genes y proteínas relacionadas con el Signo de Hoffmann, detectando grupos funcionales y patrones de interacción potencialmente implicados en procesos neuronales y enfermedades neurodegenerativas, lo que contribuye al entendimiento de sus bases moleculares.

\subsection{Conclusiones Específicas}

\begin{enumerate}
	\item Se identificaron genes relacionados con el Signo de Hoffmann utilizando la Human Phenotype Ontology (HPO) y otras bases de datos relevantes, además de genes que podrían estar indirectamente relacionadas, facilitando la observación de patrones funcionales que no eran evidentes.
	
	\item Se construyó con éxito una red de interacciones proteína-proteína (PPI) destacando nodos relevantes con alta conectividad.
	
	\item Mediante el cálculo de métricas topológicas, se identificaron proteínas clave dentro de la red con potencial importancia en la regulación de procesos neurobiológicos relacionados con la hiperreflexia y la neurodegeneración.
	
	\item El clustering permitió identificar grupos funcionales altamente conectados dentro de la red de interacciones. Estos grupos mostraron correlaciones con procesos biológicos relevantes, como la endocitosis y la regulación del citoesqueleto.
	
	\item El análisis de enriquecimiento funcional confirmó que los genes identificados están involucrados en funciones biológicas críticas, como la regulación de la biosíntesis de ATP y el transporte intracelular, y en vías relacionadas con la neurodegeneración.
\end{enumerate}

\section{Líneas futuras de investigación}

Este estudio abre varias líneas sobre los mecanismos moleculares del Signo de Hoffmann y su relación con enfermedades neurogenerativas. Sería enriquecedor validar las interacciones propuestas mediante estudios experimentales e integrar nuevos datos para un análisis más completo.

\begin{itemize}
	\item Validación experimental de las interacciones proteína-proteína identificadas en modelos celulares o animales.
	\item Exploración de las modificaciones postraduccionales de proteínas clave involucradas en los procesos identificados.
	\item Ampliación del análisis a otras patologías relacionadas con hiperreflexia para identificar posibles conexiones comunes.
	\item Análisis longitudinal de la expresión de los genes identificados en diferentes etapas de la neurodegeneración.
	\item Integración de datos ómicos adicionales (transcriptómica, epigenómica) para un análisis más exhaustivo de las redes moleculares implicadas.
\end{itemize}

