\section{Introducción}
El signo de Hoffmann es un reflejo muscular asociado a lesiones en la médula espinal, fue propuesto por primera vez por Johann Hoffmann, un neurólogo alemán, a finales del siglo XIX. Fue descrito por primera vez gracias a Hans Curschmann, uno de sus asistentes, en 1911. Como lo demuestra la revisión histórica de \cite{ BENDHEIM}. Este reflejo se trata de una flexión involuntaria del dedo pulgar y o indice al percutir suavemente el lecho ungueal del dedo medio o corazón, y su presencia puede indicar una lesión en los tractos corticoespinales, vías neuronales que conectan la corteza cerebral con la médula espinal.